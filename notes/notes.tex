\documentclass[a4paper]{article}
\usepackage[hidelinks]{hyperref}
\usepackage{amsmath}

\newcommand{\ce}{\operatorname{e}}
\newcommand{\pa}[1]{\left(#1\right)}
\newcommand{\ol}{\overline}

\title{The theory behind the amplifier}
\author{Samuel Mimram}

\begin{document}
\maketitle

This is my own take at trying to understand and implement~\cite{wright2019real}.

\section{Activation functions}
We write $\sigma$ for the \emph{sigmoid function} (aka \emph{logistic function})
defined by
\[
  \sigma(x) = \frac 1 {1 + \ce^{-x}}
\]
whose derivative is
\[
  \sigma'(x) = \sigma(x)(1-\sigma(x))
\]

We write $\phi$ for the \emph{hyperbolic tangent function} defined as
\[
  \phi(x)=\frac{\ce^{x}-\ce^{-x}}{\ce^{x}+\ce^{-x}}
\]
whose derivative is
\[
  \phi'(x)=1-(\phi(x))^2
\]

\section{Gated recurrent units}
A \emph{gated recurrent unit} or \emph{GRU} takes as input the signal $x$ and
the (hidden) state $h$ and returns an output $y$ and a new value for the state
$h$. In practice the new state $h$ and the output $y$ are the same here, so that
we reserve the notation~$h$ for the previous input state. We have
\begin{align*}
  u&=\sigma(v^ux+w^uh+b^u)&&\text{update gate}\\
  r&=\sigma(v^rx+w^rh+b^r)&&\text{reset gate}\\
  c&=\phi(v^cx+w^crh+b^c)&&\text{candidate activation}\\
  y&=(1-u)h+uc&&\text{output}
\end{align*}

The partial derivatives are
\begin{align*}
  \frac{\partial y}{\partial u}&=c-h\\
  \frac{\partial y}{\partial c}&=u\\
  \frac{\partial y}{\partial r}&=\frac{\partial y}{\partial c}\frac{\partial c}{\partial r}=u(1-c^2)w^ch\\
  \frac{\partial y}{\partial v^u}&=\frac{\partial y}{\partial u}\frac{\partial u}{\partial v^u}=(c-h)u(1-u)x\\
  \frac{\partial y}{\partial w^u}&=\frac{\partial y}{\partial u}\frac{\partial u}{\partial w^u}=(c-h)u(1-u)h\\
  \frac{\partial y}{\partial b^u}&=\frac{\partial y}{\partial u}\frac{\partial u}{\partial b^u}=(c-h)u(1-u)\\
  \frac{\partial y}{\partial v^r}&=\frac{\partial y}{\partial r}\frac{\partial r}{\partial v^r}=u(1-c^2)w^chr(1-r)x\\
  \frac{\partial y}{\partial w^r}&=\frac{\partial y}{\partial r}\frac{\partial r}{\partial w^r}=u(1-c^2)w^chr(1-r)h\\
  \frac{\partial y}{\partial b^r}&=\frac{\partial y}{\partial r}\frac{\partial r}{\partial b^r}=u(1-c^2)w^chr(1-r)\\
  \frac{\partial y}{\partial v^c}&=\frac{\partial y}{\partial c}\frac{\partial c}{\partial v^c}=u(1-c^2)x\\
  \frac{\partial y}{\partial w^c}&=\frac{\partial y}{\partial c}\frac{\partial c}{\partial w^c}=u(1-c^2)rh\\
  \frac{\partial y}{\partial b^c}&=\frac{\partial y}{\partial c}\frac{\partial c}{\partial b^c}=u(1-c^2)
\end{align*}

\section{General architecture}
We take $n$ gated recurrent units in parallel and, writing $y_i$ for the output
of the $i$-th GRU, the global output is
\[
  z=\pa{\sum_iw^z_iy_i}+b^z
\]

\section{Gradient descent}
Given an input $x$, we write $z$ for the output of our network and $\ol z$. The error is
\[
  E=\frac 12(z-\ol z)^2
\]
We have the following partial derivatives
\begin{align*}
  \frac{\partial E}{\partial b^z}&=\frac{\partial E}{\partial z}\frac{\partial z}{\partial b^z}=(z-\ol z)\\
  \frac{\partial E}{\partial w_i^z}&=\frac{\partial E}{\partial z}\frac{\partial z}{\partial w_i^z}=(z-\ol z)y_i\\
  \frac{\partial E}{\partial y_i}&=\frac{\partial E}{\partial z}\frac{\partial z}{\partial y_i}=(z-\ol z)w_i^z
\end{align*}
which allow computing all the partial derivatives when combined with the above.

\bibliographystyle{plainurl}
\bibliography{papers}
\end{document}
