\documentclass[a4paper]{article}
\usepackage[hidelinks]{hyperref}
\usepackage{amsmath}

\newcommand{\ce}{\operatorname{e}}

\title{The theory behind the amplifier}
\author{Samuel Mimram}

\begin{document}
\maketitle

This is my own take at trying to understand and implement~\cite{wright2019real}.

\section{Activation functions}
We write $\sigma$ for the \emph{sigmoid function} (aka \emph{logistic function})
defined by
\[
  \sigma(x) = \frac 1 {1 + \ce^{-x}}
\]
whose derivative is
\[
  \sigma'(x) = \sigma(x)(1-\sigma(x))
\]

We write $\phi$ for the \emph{hyperbolic tangent function} defined as
\[
  \phi(x)=\frac{\ce^{x}-\ce^{-x}}{\ce^{x}+\ce^{-x}}
\]
whose derivative is
\[
  \phi'(x)=1-(\phi(x))^2
\]

\section{Gated recurrent units}
A \emph{gated recurrent unit} or \emph{GRU} takes as input the signal $x$ and
the (hidden) state $h$ and returns an output $y$ and a new value for the state
$h$. In practice the new state $h$ and the output $y$ are the same here, so that
we reserve the notation~$h$ to the previous input state. We have
\begin{align*}
  z&=\sigma(w^zx+w^zh+b^z)&&\text{update gate}\\
  r&=\sigma(w^rx+u^rh+b^r)&&\text{reset gate}\\
  c&=\phi(w^cx+u^crh+b^c)&&\text{candidate activation}\\
  y&=(1-z)h+zc&&\text{output}
\end{align*}

\bibliographystyle{plainurl}
\bibliography{papers}
\end{document}
